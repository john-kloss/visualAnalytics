\documentclass[11pt,]{article}
\usepackage{lmodern}
\usepackage{amssymb,amsmath}
\usepackage{ifxetex,ifluatex}
\usepackage{fixltx2e} % provides \textsubscript
\ifnum 0\ifxetex 1\fi\ifluatex 1\fi=0 % if pdftex
  \usepackage[T1]{fontenc}
  \usepackage[utf8]{inputenc}
\else % if luatex or xelatex
  \ifxetex
    \usepackage{mathspec}
  \else
    \usepackage{fontspec}
  \fi
  \defaultfontfeatures{Ligatures=TeX,Scale=MatchLowercase}
\fi
% use upquote if available, for straight quotes in verbatim environments
\IfFileExists{upquote.sty}{\usepackage{upquote}}{}
% use microtype if available
\IfFileExists{microtype.sty}{%
\usepackage{microtype}
\UseMicrotypeSet[protrusion]{basicmath} % disable protrusion for tt fonts
}{}
\usepackage[margin=1in]{geometry}
\usepackage{hyperref}
\hypersetup{unicode=true,
            pdftitle={Aufgabenblatt 2},
            pdfborder={0 0 0},
            breaklinks=true}
\urlstyle{same}  % don't use monospace font for urls
\usepackage{color}
\usepackage{fancyvrb}
\newcommand{\VerbBar}{|}
\newcommand{\VERB}{\Verb[commandchars=\\\{\}]}
\DefineVerbatimEnvironment{Highlighting}{Verbatim}{commandchars=\\\{\}}
% Add ',fontsize=\small' for more characters per line
\usepackage{framed}
\definecolor{shadecolor}{RGB}{248,248,248}
\newenvironment{Shaded}{\begin{snugshade}}{\end{snugshade}}
\newcommand{\KeywordTok}[1]{\textcolor[rgb]{0.13,0.29,0.53}{\textbf{{#1}}}}
\newcommand{\DataTypeTok}[1]{\textcolor[rgb]{0.13,0.29,0.53}{{#1}}}
\newcommand{\DecValTok}[1]{\textcolor[rgb]{0.00,0.00,0.81}{{#1}}}
\newcommand{\BaseNTok}[1]{\textcolor[rgb]{0.00,0.00,0.81}{{#1}}}
\newcommand{\FloatTok}[1]{\textcolor[rgb]{0.00,0.00,0.81}{{#1}}}
\newcommand{\ConstantTok}[1]{\textcolor[rgb]{0.00,0.00,0.00}{{#1}}}
\newcommand{\CharTok}[1]{\textcolor[rgb]{0.31,0.60,0.02}{{#1}}}
\newcommand{\SpecialCharTok}[1]{\textcolor[rgb]{0.00,0.00,0.00}{{#1}}}
\newcommand{\StringTok}[1]{\textcolor[rgb]{0.31,0.60,0.02}{{#1}}}
\newcommand{\VerbatimStringTok}[1]{\textcolor[rgb]{0.31,0.60,0.02}{{#1}}}
\newcommand{\SpecialStringTok}[1]{\textcolor[rgb]{0.31,0.60,0.02}{{#1}}}
\newcommand{\ImportTok}[1]{{#1}}
\newcommand{\CommentTok}[1]{\textcolor[rgb]{0.56,0.35,0.01}{\textit{{#1}}}}
\newcommand{\DocumentationTok}[1]{\textcolor[rgb]{0.56,0.35,0.01}{\textbf{\textit{{#1}}}}}
\newcommand{\AnnotationTok}[1]{\textcolor[rgb]{0.56,0.35,0.01}{\textbf{\textit{{#1}}}}}
\newcommand{\CommentVarTok}[1]{\textcolor[rgb]{0.56,0.35,0.01}{\textbf{\textit{{#1}}}}}
\newcommand{\OtherTok}[1]{\textcolor[rgb]{0.56,0.35,0.01}{{#1}}}
\newcommand{\FunctionTok}[1]{\textcolor[rgb]{0.00,0.00,0.00}{{#1}}}
\newcommand{\VariableTok}[1]{\textcolor[rgb]{0.00,0.00,0.00}{{#1}}}
\newcommand{\ControlFlowTok}[1]{\textcolor[rgb]{0.13,0.29,0.53}{\textbf{{#1}}}}
\newcommand{\OperatorTok}[1]{\textcolor[rgb]{0.81,0.36,0.00}{\textbf{{#1}}}}
\newcommand{\BuiltInTok}[1]{{#1}}
\newcommand{\ExtensionTok}[1]{{#1}}
\newcommand{\PreprocessorTok}[1]{\textcolor[rgb]{0.56,0.35,0.01}{\textit{{#1}}}}
\newcommand{\AttributeTok}[1]{\textcolor[rgb]{0.77,0.63,0.00}{{#1}}}
\newcommand{\RegionMarkerTok}[1]{{#1}}
\newcommand{\InformationTok}[1]{\textcolor[rgb]{0.56,0.35,0.01}{\textbf{\textit{{#1}}}}}
\newcommand{\WarningTok}[1]{\textcolor[rgb]{0.56,0.35,0.01}{\textbf{\textit{{#1}}}}}
\newcommand{\AlertTok}[1]{\textcolor[rgb]{0.94,0.16,0.16}{{#1}}}
\newcommand{\ErrorTok}[1]{\textcolor[rgb]{0.64,0.00,0.00}{\textbf{{#1}}}}
\newcommand{\NormalTok}[1]{{#1}}
\usepackage{longtable,booktabs}
\usepackage{graphicx,grffile}
\makeatletter
\def\maxwidth{\ifdim\Gin@nat@width>\linewidth\linewidth\else\Gin@nat@width\fi}
\def\maxheight{\ifdim\Gin@nat@height>\textheight\textheight\else\Gin@nat@height\fi}
\makeatother
% Scale images if necessary, so that they will not overflow the page
% margins by default, and it is still possible to overwrite the defaults
% using explicit options in \includegraphics[width, height, ...]{}
\setkeys{Gin}{width=\maxwidth,height=\maxheight,keepaspectratio}
\IfFileExists{parskip.sty}{%
\usepackage{parskip}
}{% else
\setlength{\parindent}{0pt}
\setlength{\parskip}{6pt plus 2pt minus 1pt}
}
\setlength{\emergencystretch}{3em}  % prevent overfull lines
\providecommand{\tightlist}{%
  \setlength{\itemsep}{0pt}\setlength{\parskip}{0pt}}
\setcounter{secnumdepth}{0}
% Redefines (sub)paragraphs to behave more like sections
\ifx\paragraph\undefined\else
\let\oldparagraph\paragraph
\renewcommand{\paragraph}[1]{\oldparagraph{#1}\mbox{}}
\fi
\ifx\subparagraph\undefined\else
\let\oldsubparagraph\subparagraph
\renewcommand{\subparagraph}[1]{\oldsubparagraph{#1}\mbox{}}
\fi

%%% Use protect on footnotes to avoid problems with footnotes in titles
\let\rmarkdownfootnote\footnote%
\def\footnote{\protect\rmarkdownfootnote}

%%% Change title format to be more compact
\usepackage{titling}

% Create subtitle command for use in maketitle
\newcommand{\subtitle}[1]{
  \posttitle{
    \begin{center}\large#1\end{center}
    }
}

\setlength{\droptitle}{-2em}
  \title{Aufgabenblatt 2}
  \pretitle{\vspace{\droptitle}\centering\huge}
  \posttitle{\par}
  \author{}
  \preauthor{}\postauthor{}
  \date{}
  \predate{}\postdate{}

\usepackage[german]{babel}

\begin{document}
\maketitle

\begin{enumerate}
\def\labelenumi{\arabic{enumi}.}
\tightlist
\item
  Gegeben sei der nachstehende zweidimensionale Datensatz. Führen Sie
  ein \(K\)-means Clustering mit \(K=3\) unter Verwendung der
  euklidischen Distanz durch. Verwenden Sie die ersten drei Punkte als
  Anfangszentroiden. Geben Sie bei jeder Algorithmeniteration jeweils
  die Distanzen zwischen Zentroiden und allen Punkten an und berechnen
  Sie nach jeder Neuzuordnung der Punkte die veränderten Zentroiden.
\end{enumerate}

\begin{longtable}[]{@{}lllllllllllll@{}}
\toprule
& p1 & p2 & p3 & p4 & p5 & p6 & p7 & p8 & p9 & p10 & p11 &
p12\tabularnewline
\midrule
\endhead
x & 2.0 & 2.0 & 2.0 & 2.5 & 2.5 & 3.0 & 4.0 & 4.0 & 4.5 & 4.5 & 4.5 &
4.5\tabularnewline
y & 1.0 & 1.5 & 2.0 & 1.0 & 2.0 & 4.0 & 1.0 & 2.5 & 1.0 & 1.5 & 2.5 &
3.0\tabularnewline
\bottomrule
\end{longtable}

\begin{Shaded}
\begin{Highlighting}[]
 \NormalTok{dat <-}\StringTok{ }\KeywordTok{tibble}\NormalTok{(}
   \DataTypeTok{x =} \KeywordTok{c}\NormalTok{(}\FloatTok{2.0}\NormalTok{, }\FloatTok{2.0}\NormalTok{, }\FloatTok{2.0}\NormalTok{, }\FloatTok{2.5}\NormalTok{, }\FloatTok{2.5}\NormalTok{, }\FloatTok{3.0}\NormalTok{, }\FloatTok{4.0}\NormalTok{, }\FloatTok{4.0}\NormalTok{, }\FloatTok{4.5}\NormalTok{, }\FloatTok{4.5}\NormalTok{, }\FloatTok{4.5} \NormalTok{, }\FloatTok{4.5}\NormalTok{),}
   \DataTypeTok{y =} \KeywordTok{c}\NormalTok{(}\FloatTok{1.0}\NormalTok{, }\FloatTok{1.5}\NormalTok{, }\FloatTok{2.0}\NormalTok{, }\FloatTok{1.0}\NormalTok{, }\FloatTok{2.0}\NormalTok{, }\FloatTok{4.0}\NormalTok{, }\FloatTok{1.0}\NormalTok{, }\FloatTok{2.5}\NormalTok{, }\FloatTok{1.0}\NormalTok{, }\FloatTok{1.5}\NormalTok{, }\FloatTok{2.5} \NormalTok{, }\FloatTok{3.0}\NormalTok{)}
 \NormalTok{)}

\CommentTok{# Lösung zu Aufgabe 1...}

\CommentTok{#create inital centers}
\NormalTok{centers<-}\StringTok{ }\NormalTok{dat[}\DecValTok{1}\NormalTok{:}\DecValTok{3}\NormalTok{,}\DecValTok{1}\NormalTok{:}\DecValTok{2}\NormalTok{] }
\NormalTok{curCenters <-}\StringTok{ }\NormalTok{dat[}\DecValTok{2}\NormalTok{:}\DecValTok{4}\NormalTok{,}\DecValTok{1}\NormalTok{:}\DecValTok{2}\NormalTok{]}

\NormalTok{while(!}\KeywordTok{isTRUE}\NormalTok{(}\KeywordTok{all.equal}\NormalTok{(curCenters, centers)))\{}
  \NormalTok{curCenters <-}\StringTok{ }\NormalTok{centers}
  \KeywordTok{print}\NormalTok{(}\KeywordTok{rdist}\NormalTok{(centers, dat))}
  \NormalTok{dat <-}\StringTok{ }\KeywordTok{mutate}\NormalTok{(dat,}\DataTypeTok{cluster=}\NormalTok{(}\KeywordTok{apply}\NormalTok{(}\KeywordTok{rdist}\NormalTok{(centers, dat),}\DecValTok{2}\NormalTok{,which.min)))}
  \NormalTok{for(i in }\DecValTok{1}\NormalTok{:}\DecValTok{3}\NormalTok{)\{}
    \NormalTok{current <-}\StringTok{ }\KeywordTok{filter}\NormalTok{(dat, cluster==i)}
    \NormalTok{centers[i,}\DecValTok{1}\NormalTok{] <-}\StringTok{ }\KeywordTok{sum}\NormalTok{(current$x)/}\KeywordTok{nrow}\NormalTok{(current)}
    \NormalTok{centers[i,}\DecValTok{2}\NormalTok{] <-}\StringTok{ }\KeywordTok{sum}\NormalTok{(current$y)/}\KeywordTok{nrow}\NormalTok{(current)}
  \NormalTok{\}}
\NormalTok{\}}
\end{Highlighting}
\end{Shaded}

\begin{verbatim}
##      [,1] [,2] [,3]      [,4]      [,5]     [,6]     [,7]     [,8]
## [1,]  0.0  0.5  1.0 0.5000000 1.1180340 3.162278 2.000000 2.500000
## [2,]  0.5  0.0  0.5 0.7071068 0.7071068 2.692582 2.061553 2.236068
## [3,]  1.0  0.5  0.0 1.1180340 0.5000000 2.236068 2.236068 2.061553
##          [,9]   [,10]    [,11]    [,12]
## [1,] 2.500000 2.54951 2.915476 3.201562
## [2,] 2.549510 2.50000 2.692582 2.915476
## [3,] 2.692582 2.54951 2.549510 2.692582
##          [,1]     [,2]     [,3]      [,4]      [,5]     [,6]      [,7]
## [1,] 1.250000 1.346291 1.600781 0.7500000 1.2500000 3.010399 0.7500000
## [2,] 1.346291 1.250000 1.346291 0.9013878 0.9013878 2.512469 0.9013878
## [3,] 2.187401 1.835226 1.565691 1.9021187 1.1334559 1.396921 1.7658017
##           [,8]     [,9]    [,10]    [,11]    [,12]
## [1,] 1.6770510 1.250000 1.346291 1.952562 2.358495
## [2,] 1.2500000 1.346291 1.250000 1.600781 1.952562
## [3,] 0.6066758 1.987810 1.592081 1.096079 1.133456
##          [,1]      [,2]      [,3]      [,4]      [,5]     [,6]     [,7]
## [1,] 1.250000 1.3462912 1.6007811 0.7500000 1.2500000 3.010399 0.750000
## [2,] 1.060660 0.7905694 0.7905694 0.7905694 0.3535534 2.263846 1.457738
## [3,] 2.828427 2.5000000 2.2360680 2.5000000 1.8027756 1.414214 2.000000
##          [,8]     [,9]    [,10]     [,11]    [,12]
## [1,] 1.677051 1.250000 1.346291 1.9525624 2.358495
## [2,] 1.457738 1.903943 1.767767 1.9039433 2.150581
## [3,] 0.500000 2.061553 1.581139 0.7071068 0.500000
##           [,1]      [,2]      [,3]     [,4]      [,5]     [,6]      [,7]
## [1,] 1.8791620 1.9121323 2.0691182 1.380670 1.6298006 3.005204 0.1767767
## [2,] 0.6373774 0.1767767 0.3952847 0.728869 0.5303301 2.531057 1.9764235
## [3,] 2.8284271 2.5000000 2.2360680 2.500000 1.8027756 1.414214 2.0000000
##          [,8]      [,9]    [,10]     [,11]    [,12]
## [1,] 1.380670 0.6373774 0.728869 1.5103807 1.976424
## [2,] 2.069118 2.4558603 2.378287 2.5310571 2.744312
## [3,] 0.500000 2.0615528 1.581139 0.7071068 0.500000
##           [,1]     [,2]      [,3]      [,4]      [,5]     [,6]     [,7]
## [1,] 2.3392781 2.357023 2.4776781 1.8408935 2.0138410 3.131382 0.372678
## [2,] 0.5385165 0.200000 0.5385165 0.5830952 0.5830952 2.624881 1.868154
## [3,] 2.8284271 2.500000 2.2360680 2.5000000 1.8027756 1.414214 2.000000
##          [,8]      [,9]    [,10]     [,11]    [,12]
## [1,] 1.374369 0.2357023 0.372678 1.3437096 1.840894
## [2,] 2.059126 2.3537205 2.300000 2.5079872 2.745906
## [3,] 0.500000 2.0615528 1.581139 0.7071068 0.500000
\end{verbatim}

\begin{Shaded}
\begin{Highlighting}[]
\KeywordTok{ggplot}\NormalTok{(dat, }\KeywordTok{aes}\NormalTok{(}\DataTypeTok{x=}\NormalTok{x, }\DataTypeTok{y=}\NormalTok{y)) +}\StringTok{ }\KeywordTok{geom_point}\NormalTok{( }\KeywordTok{aes}\NormalTok{(}\DataTypeTok{colour=}\NormalTok{cluster), }\DataTypeTok{size=}\DecValTok{3}\NormalTok{) +}\StringTok{  }\KeywordTok{scale_colour_gradientn}\NormalTok{(}\DataTypeTok{colours=}\KeywordTok{rainbow}\NormalTok{(}\DecValTok{4}\NormalTok{))}
\end{Highlighting}
\end{Shaded}

\includegraphics{ex_02_solution_template_files/figure-latex/unnamed-chunk-1-1.pdf}

\begin{enumerate}
\def\labelenumi{\arabic{enumi}.}
\setcounter{enumi}{1}
\tightlist
\item
  Eine Schule möchte ihre Schüler nach den Leistungen bei zwei
  Zwischenprüfungen gruppieren. Es wird davon ausgegangen, dass es
  mindestens 2 Cluster von Schülern gibt. Laden Sie die Datei
  \texttt{clustering-student-mat.csv} ein. Die Datei enthält zu jeder
  der beiden Prüfungen die Anzahl der erzielten Punktzahl für
  insgesamt 395 Schüler.\\
  Führen Sie je ein \(K\)-means-Clustering für alle
  \(k\in \{2,3,\ldots,8\}\) durch. Stellen Sie die Clusterzuordnungen
  der Punkte in einem Streudiagramm (Scatter Plot) dar.
\end{enumerate}

\begin{Shaded}
\begin{Highlighting}[]
\CommentTok{# Lösung zu Aufgabe 2...}
 \NormalTok{student <-}\StringTok{ }\KeywordTok{read_csv}\NormalTok{(}\KeywordTok{str_c}\NormalTok{(}\KeywordTok{dirname}\NormalTok{(}\KeywordTok{getwd}\NormalTok{()), }\StringTok{"/visualAnalytics/Data/clustering-student-mat.csv"}\NormalTok{))}
\NormalTok{k<-}\KeywordTok{c}\NormalTok{(}\DecValTok{2}\NormalTok{:}\DecValTok{8}\NormalTok{)}
\NormalTok{for (j in k)\{}
  \NormalTok{clust<-}\KeywordTok{kmeans}\NormalTok{(student,}\DataTypeTok{centers=}\NormalTok{j)}
  \NormalTok{student<-student %>%}\StringTok{ }\KeywordTok{mutate}\NormalTok{(}\DataTypeTok{cluster =} \KeywordTok{factor}\NormalTok{(clust$cluster))}
  \NormalTok{plt<-}\KeywordTok{ggplot}\NormalTok{(student, }\KeywordTok{aes}\NormalTok{(Exam1, Exam2, }\DataTypeTok{color =} \NormalTok{cluster, }
                                         \DataTypeTok{fill =} \NormalTok{cluster))+}
\StringTok{    }\KeywordTok{geom_point}\NormalTok{()}
  \KeywordTok{print}\NormalTok{(plt)}
\NormalTok{\}}
\end{Highlighting}
\end{Shaded}

\includegraphics{ex_02_solution_template_files/figure-latex/unnamed-chunk-2-1.pdf}
\includegraphics{ex_02_solution_template_files/figure-latex/unnamed-chunk-2-2.pdf}
\includegraphics{ex_02_solution_template_files/figure-latex/unnamed-chunk-2-3.pdf}
\includegraphics{ex_02_solution_template_files/figure-latex/unnamed-chunk-2-4.pdf}
\includegraphics{ex_02_solution_template_files/figure-latex/unnamed-chunk-2-5.pdf}
\includegraphics{ex_02_solution_template_files/figure-latex/unnamed-chunk-2-6.pdf}
\includegraphics{ex_02_solution_template_files/figure-latex/unnamed-chunk-2-7.pdf}

\begin{enumerate}
\def\labelenumi{\arabic{enumi}.}
\setcounter{enumi}{2}
\tightlist
\item
  Ermitteln Sie für das Clustering aus Ausgabe 2 den optimalen Wert
  für die Anzahl der Cluster \(K\) mithilfe des
  Silhouetten-Koeffizienten. Bewerten Sie das Ergebnis im Hinblick auf
  die Repräsentativität der Zentroiden bezüglich ihres Clusters.
\end{enumerate}

\begin{Shaded}
\begin{Highlighting}[]
\CommentTok{# Lösung zu Aufgabe 3...}
\KeywordTok{library}\NormalTok{(cluster)}
\NormalTok{k<-}\KeywordTok{c}\NormalTok{(}\DecValTok{2}\NormalTok{:}\DecValTok{8}\NormalTok{)}
\NormalTok{for (j in k)\{}
  \NormalTok{clust<-}\KeywordTok{kmeans}\NormalTok{(student,}\DataTypeTok{centers=}\NormalTok{j)}
  \NormalTok{student<-student %>%}\StringTok{ }\KeywordTok{mutate}\NormalTok{(}\DataTypeTok{cluster =} \KeywordTok{factor}\NormalTok{(clust$cluster))}
  \NormalTok{si <-}\StringTok{ }\KeywordTok{silhouette}\NormalTok{(clust$cluster, }\KeywordTok{dist}\NormalTok{(student))}
  \KeywordTok{window}\NormalTok{(si)}
  \KeywordTok{pdf}\NormalTok{(}\KeywordTok{paste0}\NormalTok{(j, }\StringTok{'plot.pdf'}\NormalTok{,}\DataTypeTok{sep=}\StringTok{""}\NormalTok{))}
  \KeywordTok{plot}\NormalTok{(si)}
  \KeywordTok{dev.off}\NormalTok{()}
\NormalTok{\}}
\CommentTok{#Repräsentativität der Zentroiden zu ihrem Cluster = Silhouetten-Koeffizienten sollten ähnlich sein}
\CommentTok{#2 bis 5 relativ durchwachsen, viele Mitglieder eines Clusters haben nahezu gleiche Entfernung zu anderen Clusterelementen wie geringste Entfernung zu Element aus anderem Cluster, die Werte meist im mittleren Bereich ca. um .50 herum -> mittelgute Repräsentativität der Cluster, auch negativ-Werte dabei -> mögliche Falsch-Klassifizierung}
\CommentTok{#ab 6 stärkere Repräsentativität der Zentroiden, da einzelne Durchschnittswerte für ein Cluster höher werden und Durchwachsenheit der Werte im Cluster abnimmt}
\CommentTok{# 8 beste Repräsentativität, aufgrund der ähnlichen Silhouettenwerte im jeweiligen Cluster}
\end{Highlighting}
\end{Shaded}

\begin{enumerate}
\def\labelenumi{\arabic{enumi}.}
\setcounter{enumi}{3}
\tightlist
\item
  Gegeben sei die nachstehende Distanzmatrix. Führen Sie agglomeratives
  hierarchisches Clustering mit \emph{single} und \emph{complete}
  Linkage durch. Stellen Sie das Ergebnis in einem Dendrogramm dar. Das
  Dendrogramm sollte die Reihenfolge des Zusammenfügens der Punkte
  darstellen.
\end{enumerate}

\begin{Shaded}
\begin{Highlighting}[]
\NormalTok{dm <-}\StringTok{ }\KeywordTok{tribble}\NormalTok{(~p1,~p2,~p3,~p4,~p5,}
              \FloatTok{0.00}\NormalTok{, }\FloatTok{0.02}\NormalTok{, }\FloatTok{0.90}\NormalTok{, }\FloatTok{0.36}\NormalTok{, }\FloatTok{0.53}\NormalTok{,}
              \FloatTok{0.02}\NormalTok{, }\FloatTok{0.00}\NormalTok{, }\FloatTok{0.65}\NormalTok{, }\FloatTok{0.15}\NormalTok{, }\FloatTok{0.24}\NormalTok{,}
              \FloatTok{0.90}\NormalTok{, }\FloatTok{0.65}\NormalTok{, }\FloatTok{0.00}\NormalTok{, }\FloatTok{0.59}\NormalTok{, }\FloatTok{0.45}\NormalTok{,}
              \FloatTok{0.36}\NormalTok{, }\FloatTok{0.15}\NormalTok{, }\FloatTok{0.59}\NormalTok{, }\FloatTok{0.00}\NormalTok{, }\FloatTok{0.56}\NormalTok{,}
              \FloatTok{0.53}\NormalTok{, }\FloatTok{0.24}\NormalTok{, }\FloatTok{0.45}\NormalTok{, }\FloatTok{0.56}\NormalTok{, }\FloatTok{0.00}\NormalTok{) %>%}\StringTok{ }\KeywordTok{as.matrix}\NormalTok{()}
\KeywordTok{rownames}\NormalTok{(dm) <-}\StringTok{ }\NormalTok{letters[}\DecValTok{1}\NormalTok{:}\DecValTok{5}\NormalTok{]}
\KeywordTok{colnames}\NormalTok{(dm) <-}\StringTok{ }\NormalTok{letters[}\DecValTok{1}\NormalTok{:}\DecValTok{5}\NormalTok{]}
\NormalTok{knitr::}\KeywordTok{kable}\NormalTok{(dm)}
\end{Highlighting}
\end{Shaded}

\begin{longtable}[]{@{}lrrrrr@{}}
\toprule
& a & b & c & d & e\tabularnewline
\midrule
\endhead
a & 0.00 & 0.02 & 0.90 & 0.36 & 0.53\tabularnewline
b & 0.02 & 0.00 & 0.65 & 0.15 & 0.24\tabularnewline
c & 0.90 & 0.65 & 0.00 & 0.59 & 0.45\tabularnewline
d & 0.36 & 0.15 & 0.59 & 0.00 & 0.56\tabularnewline
e & 0.53 & 0.24 & 0.45 & 0.56 & 0.00\tabularnewline
\bottomrule
\end{longtable}

\begin{Shaded}
\begin{Highlighting}[]
\CommentTok{# Lösung zu Aufgabe 4...}


\NormalTok{hc <-}\StringTok{ }\KeywordTok{hclust}\NormalTok{(}\KeywordTok{as.dist}\NormalTok{(dm), }\DataTypeTok{method =} \StringTok{"single"}\NormalTok{)}
\NormalTok{d_sl <-}\StringTok{ }\KeywordTok{ggdendrogram}\NormalTok{(hc, }\DataTypeTok{rotate =} \NormalTok{F, }\DataTypeTok{size =} \DecValTok{2}\NormalTok{)}
\NormalTok{d_sl}
\end{Highlighting}
\end{Shaded}

\includegraphics{ex_02_solution_template_files/figure-latex/unnamed-chunk-4-1.pdf}

\begin{center}\rule{0.5\linewidth}{\linethickness}\end{center}

Datensatz für Aufgabe 2:\\
\url{http://isgwww.cs.uni-magdeburg.de/cv/lehre/VisAnalytics/material/exercise/datasets/clustering-student-mat.csv}


\end{document}
